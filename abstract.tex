% $Log: abstract.tex,v $
%             
%              ABSTRACT
% 
%

Around the world, efforts are underway to commission several kilometer-scale 
laser interferometers to detect gravitational radiation. In the United States,
there are two collocated interferometers in Hanford, Washington and one interferometer
in Livingston, Louisiana. Together, these three interferometers form the Laser
Interferometric Gravitational-wave Observatory (LIGO). 

The core of the work described in this thesis is the modeling and reduction of
the noise in the interferometers which limits their ultimate sensitivity.

A vital component of the noise reduction is the modeling, design, and implementation
of $\sim$100 feedback control systems. The most critical of these systems
are described and motivated.

Although improvements are continuously being made to the stability and noise character
of these detectors, several months of data have been collected. Various efforts
are underway to search through these data for gravitational wave signals.
Included here, is a description of a search made through the data for signals
from the ringdown of the quasi-normal modes of Kerr black holes.

In addition, several possible future improvements to the detectors are outlined.
