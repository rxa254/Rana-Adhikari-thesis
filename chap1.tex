%---   CHAPTER 1          - ---- ---- -- -----------------

%---      GRAVITY WAVES           --------------------------
%

\chapter{Gravitational Radiation}
\label{chap:GW}

\begin{figure}[!h]
\centerline{\includegraphics[angle=0,width=6.5in]{Figures/Chap1/GRB-DestroyStar.jpg}}
\end{figure}
\clearpage

This chapter describes gravitational waves and their possible sources.

Section~\ref{sec:GR} describes the concept of gravitational waves and the space-time
strain which we expect to measure in the far-field of a radiator.

Section~\ref{sec:Sources} discusses 4 different classes of signals and is a review
of current source strength and rate estimates.

%Section~\ref{sec:Searches} lists selected gravitational wave searches done in the
%past and tabulates the resulting upper limits.


%------------------------------------------------------------------------------
\section{Gravitational Radiation in General Relativity}
\label{sec:GR}

%In Special Relativity, the laws of physics are the same to observers in
%different inertial reference frames. The space-time interval between events
%is invariant among these reference frames. The interval is defined as:

%\begin{equation}
%ds^2 \equiv -c^2 dt^2 + dx^2 + dy^2 + dz^2 \equiv  \eta_{\mu \nu} dx^{\mu} dx^{\nu} 
%\label{eq:special}
%\end{equation}

%Here the $\eta$ tensor defines the Minkowski metric.

The theory of General Relativity~\cite{Einstein:Book} describes gravity as a 
consequence of the curvature of space and time (or space-time). One of the
predictions of the theory is gravitational radiation from fluctuating
mass-energy distributions~\cite{MTW}. Although these ripples can, 
in principle, severely distort the space and time very near the radiator,
far from the source one can express the effect of these waves as small
perturbations to the otherwise flat space-time background. In this weak field limit
the space-time metric can be approximated as

\begin{equation}
g_{\mu \nu} \simeq \eta_{\mu \nu} + h_{\mu \nu}, 
                  \quad \mbox{where} \quad   |h_{\mu \nu}| \ll 1
\label{eq:gmunu}
\end{equation}
and where $\eta_{\mu \nu}$ is the Minkowski metric representing flat space and 
$h_{\mu \nu}$ is the perturbation to flat space due to the gravity wave.

By an adept coordinate transform~\cite{MTW} the gravitational wave may be written as:

\begin{equation}
h_{\mu \nu}(z,t) = 
\left(
\begin{matrix}
0    & 0        & 0          & 0 \\
0    & -h_{+}    &  h_{\times}     & 0 \\
0    & h_{\times}    & h_{+}     & 0 \\
0    & 0        & 0          & 0 
\end{matrix} 
\right) 
\cos{\omega \left( \frac{z}{c}- t \right)}
\label{eq:hmatrix}
\end{equation}
where $\omega$ is the gravitational wave frequency and the two independent 
polarizations are $h_{+}$ and $h_{\times}$.



\subsection{Measurable Effect on Free Masses}

From an observers point of view, we can ask about what the measurable effects are
of gravitational waves. To answer this we set up two free masses, one located at
the origin and one located a distance, $x = L$, from the origin. We can measure the
separation between these two masses by sending a plane wave of light from the origin
to bounce off of the far mass and measure the phase of the return wave.
The accrued round trip phase is:

\begin{equation}
\Phi_{rt}(t_{rt}) = \int\limits_{0}^{t_{rt}} 2 \pi f \, dt
\label{eq:roundtripflat}
\end{equation}
where $t_{rt}$ is the time it takes for the light to make one round trip and
$f$ is the frequency of the light. In the absence of radiation, we can do the
integral by changing it into an integral over length. To do this we use the 
flat space metric, $\eta_{\mu \nu}$, to relate space and time for light
($t_{rt} = 2 L/c$ and $dt = dx / c$). 

In the presence of a gravitational wave, we instead use Equation~\ref{eq:gmunu} to calculate
the space-time interval so now the round trip phase is

\begin{equation}
\Phi_{rt}(t_{rt}) = 2 \frac{2 \pi f}{c} \int\limits_{0}^{L} \sqrt{|g_{xx}|} \, dx
                    \simeq 2 (1 - h_{+}/2) \frac{2 \pi L}{\lambda}
\label{eq:roundtrip}
\end{equation}
in the case of a ''plus'' oriented wave with a period much longer than the round trip
light travel time. Repeating this integral, but doing the
integration now along the y-axis, we get that 
$\Phi_{rt} \simeq 2 (1 + h_{+}/2) (2 \pi L / \lambda)$. The difference
in the phase shift between the two arms gives 
$\Delta \Phi \simeq 2 h_{+} (2 \pi L / \lambda)$. 

Interpreting the phase shifts as length measurements indicates that the apparent 
length of each arm is stretched and compressed as the gravity wave passes. 
A diagram of this is shown in Figure~\ref{fig:CropCircles}. The
length shift is proportional to the original distance between the masses,

\begin{equation}
\frac{\Delta L}{L} = \frac{1}{2} h_+
\label{eq:h}
\end{equation}
which is why a gravitational wave is usually said to cause a strain in space.


\begin{figure}[!h]
\centerline{\includegraphics[angle=0,width=6.5in]{Figures/Chap1/CropCircles.png}}
\caption[Effect of gravitational waves on Test particles]{Shown are the effects of $+$ and
         $\times$ waves propagating in the z direction on a circle of test particles
         in the x-y plane.}
\label{fig:CropCircles}
\end{figure}


\subsection{Radiation Amplitudes}

Conservation of mass-energy, linear momentum, and angular momentum rule out 
monopole, dipole, and ''magnetic'' dipole radiation, respectively. With no
conservation law to rule it out, the leading term in gravitational
radiation is the oscillating quadrupole mass-energy distribution. In
addition, for the wave to carry away energy from the source, the amplitude
of the wave must decay as $\sim 1/r$.

A rough estimate for the strain amplitude is\cite{Kip:300}:

\begin{equation}
\begin{aligned}
h &\sim \frac{G}{c^4}\frac{\ddot{Q}}{r} \\
  &\sim \frac{G}{c^4}\frac{E_{kin}^{ns}}{r} \\
  &\sim 10^{-19} \left(\frac{E_{kin}^{ns}}{M_{\astrosun} c^2}\right)
                    \left(\frac{1 \, \mbox{Mpc}}{r}\right)
\end{aligned}
\label{eq:h_estimate}
\end{equation}
using the estimate that the non-spherical kinetic energy, $E_{kin}^{ns}$,
contributing to gravitational radiation is roughly equal to the second
time derivative of the quadrupole moment, $\ddot{Q}$. This is a very
optimistic estimate assuming a huge amount of energy is converted into
gravitational waves in a neighboring galaxy. However,
it indicates an upper limit to expected signal strengths. 

The radiated energy (luminosity) is related to the strain by~\cite{Landau:Fields}:

\begin{equation}
\begin{aligned}
\frac{dE_{\mbox{\tiny GW}}}{dt} &= \frac{c^3 r^2}{4 \, G} 
                      (\dot{h_{+}^2} + \dot{h_{\times}^2}) \\
                   &\simeq 10^{34} \left(\frac{r}{1 \, \mbox{Mpc}}\right)^2
                                   \left(\frac{|h|}{10^{-23}}\right)^2 
                     \mbox{Watts}
\label{eq:Luminosity}
\end{aligned}
\end{equation}
This estimate gives a very small measurable strain, even though the radiated
energy is quite large; space-time is a very 'stiff' wave medium.


%------------------------------------------------------------------------------
\section{Astrophysical Sources}
\label{sec:Sources}

The following sections briefly describe some types of gravitational wave
sources and their predicted strengths, frequencies, and detection rates. 
A more extensive survey can be found in Refs.~\cite{Kip:300,Kip:Probing}

Later in the thesis (Chapter~\ref{chap:ringdowns}), a search for ringdowns of
black hole quasi-normal modes is described. To accompany that work, the end of 
this chapter describes the generation of ringdown signals.


\subsection{Monochromatic Signals}

\begin{figure}[!h]
\centerline{\includegraphics[angle=0,width=6.5in]{Figures/Chap1/pulsars.pdf}}
\caption[Known Pulsars]{Upper limits on the amplitudes of many known
         pulsars compared to the upper limit set by each 
         interferometer separately during the second LIGO science run (S2). The pulsar 
         amplitude limits are made by assuming that all of the
         rotational energy loss of the pulsar goes into gravitational radiation.
         The 'SRD' curve is the LIGO Science Requirement for a 4 km interferometer
         after 1 year of integration.}
\label{fig:pulsars}
\end{figure}

One class of signal being searched for emits radiation at a single frequency,
producing a long continuous wave in the source's reference frame. 
The most commonly described
monochromatic  source is the radiation from a non-axisymmetric pulsar.
The time-dependent quadrupole moment necessary to generate gravitational waves
may come from a wobbling rotation (spin axis not aligned with a principle axis)
or a small deviation of the pulsar shape from perfect axial symmetry (a bump).
In the latter case the gravitational wave strain can be written as~\cite{Kip:300}

\begin{equation}
h \sim 2 \times 10^{-26} \left(\frac{f_{rot}}{1 \, \mbox{kHz}} \right)^2
             \left(\frac{10 \, \mbox{kpc}}{r} \right)
             \left(\frac{\epsilon}{10^{-6}} \right)
\end{equation}
where $f_{rot}$ is the frequency of rotation, $r$ is the
distance between the source and the detector, and 
$\epsilon \equiv (I_{xx}-I_{yy})/I_{zz}$, is the equatorial ellipticity.


It has been suggested~\cite{Bildsten:Braking} and somewhat supported by
observation~\cite{Deepto:Braking} that low-mass X-ray binaries (LMXBs) reach
an equilibrium where the spin-up torque due to accretion is balanced by the
spin-down from gravitational wave emission.

%The strong frequency dependence of the torque is apparently why the observed
%range of pulsar frequencies is so small (f $\approx$ 200-50 MHz crap).


\subsubsection{Upper Limits and Measurements}

Searches have been made for gravitational wave signals from pulsars; see for 
example~\cite{Hereld:Thesis,Mike:Thesis,S1:Pulsar} and references therein.

Figure~\ref{fig:pulsars} shows upper limits on the amplitudes of many known
pulsars compared to the upper limit set by each interferometer separately during
the S2 run. The amplitude of the dots are calculated by assuming that all of the
rotational energy loss of the pulsar, determined by measuring the spin-down
rate, goes into gravitational radiation.





\subsection{Stochastic Background}

Quite different in character from monochromatic sources is the
stochastic background of gravitational radiation~\cite{S1:Stochastic}.
A stochastic background can have both cosmological and
astrophysical sources such as amplification by inflation of zero-point 
metric fluctuations, phase transitions in the 
early universe, cosmic strings, and a large number of unresolved foreground
sources such as binaries and supernovae~\cite{Bruce:Houches}.

Schemes for detecting a stochastic background generally involve
cross-correlating the output of two or more 
detectors~\cite{Nelson:Thesis,Nelson:SB,Bruce:Houches,S1:Stochastic}. 
%The idea being that the there are no non-gravitational wave correlations
%among the detectors. 

The past and present analyses of a stochastic background have made some assumptions about
the statistical character of the signal: it is isotropic, unpolarized,
stationary, and Gaussian. These assumptions are discussed 
by Allen~\cite{Bruce:Stochastic};

See \cite{S1:Stochastic} for a review of current upper limits and
prospects for the future.



\subsection{Bursts}

A very large class of events are the unmodeled transients, or bursts. These
are searched for quite differently than most of the other types of signals
in this chapter. Most approaches involve looking for excess power in many
narrow bands.

Some examples of the anticipated types of burst events being searched for are
asymmetrical core collapse in supernovae, coalescence and merger of intermediate
mass black holes, and most interestingly, the unknown.


\subsubsection{Previous Searches}

A review of burst searches made with resonant bar detectors is described
elsewhere~\cite{Bars:Status,Tyson:1982}. Here I list past searches for
gravitational wave bursts using laser interferometers and the strain
amplitudes they were sensitive to:


\begin{itemize}

\item R. Forward \qquad Malibu, CA  \quad 1977  \quad h $>$ 10$^{-14}$ for 150 hours.

\item D. Dewey   \qquad MIT 1.5m \qquad 1985 \quad h $>$ 10$^{-13}$ 

\item Glasgow / Max Planck \qquad \quad 1989 \quad h $> 5 \times 10^{-16}$

\item LIGO/GEO S1 (Aug. 2002, 17 days) \quad  h $>$ 10$^{-18}$

\item LIGO/GEO S2 (Feb. 2003, 2 months) \quad  h $>$ 10$^{-19}$

\end{itemize}


\subsection{Binary Inspiral}

An extensively studied source of gravitational wave is the decaying orbit of
two compact objects, usually referred to as a binary inspiral. In the LIGO band,
these objects can be neutron stars and/or black holes (NS/NS, BH/NS, BH/BH).

The waveforms, from the NS/NS inspiral,
are believed to be sufficiently well modeled that one can search for these 
signals using a matched filter technique~\cite{Bruce:chisquare}. The BH/BH waveforms 
are much more difficult to calculate~\cite{BCV1} and there is not as much 
confidence in these waveforms. Nevertheless, a matched filter search for these 
signals is currently being pursued as well.

As the
orbit of the two bodies progresses in time, the orbital period and separation
decrease due to energy loss through gravitational radiation. As the orbital 
separation decreases, the amplitude and frequency of the signal increase until
the binary separation falls below the Innermost Stable Circular Orbit (ISCO)
and the two stars plunge together and merge.

Reconstructing the merger of two neutron stars or two black holes is a very
computationally intensive exercise and there are numerous, highly active
efforts to calculate these dynamics and the associated gravitational radiation
waveforms using fast supercomputers~\cite{Jorge:Numerical}.

\subsubsection{Rate Estimates}

The recent discovery of PSR J0737-3039~\cite{Parkes:Nature}, a highly relativistic 
binary pulsar,
increased the predicted rate of galactic NS/NS inspirals detectable by LIGO
from one every few decades to one every few years~\cite{Vicki:Rates}. 
Previous estimates of
merger rates were dominated by the parameters of the famous 
PSR B1913+16 \cite{Hulse:Pulsar,Taylor:1913}. This new binary (actually the
first detected double pulsar system) has a 3X shorter coalescence time and
a 7X smaller luminosity. These factors have radically changed the population
and merger rate estimates for NS/NS binaries in the galaxy.

Although, at the time of this writing, there have been 
7 double neutron star systems discovered~\cite{Taylor:1829}, the 
detection rate estimates for LIGO are 
still precariously dependent on the tightest, darkest binary.


\subsubsection{Upper Limits and Measurements}

A few searches have been made so far for the signatures from NS/NS inspiral
events~\cite{S1:Inspiral,40m:Inspiral,TAMA:Inspiral}. No detections have been
claimed yet. The stated upper limits for NS/NS inspirals in the galaxy are:

\begin{itemize}
\item Caltech 40m  4400 / year
\item TAMA DT6  5000 / year (within 6 kpc) 
\item LIGO S1  170 /  year
\item LIGO S2  50 / year
\end{itemize}
The Caltech 40m upper limit comes from a short run made in November of 1994
using the 40 m prototype in Pasadena, CA in a non-recycled, non-optically
recombined state. The TAMA DT6 (6th Data Taking run) data is from a 
recombined but not power recycled
300 m interferometer. Both the LIGO S1 \& S2 data were taken with power-recycled
interferometers operating in coincidence.


\subsection{Ringdowns}
\label{sec:Ringdowns}

There are three distinct phases in the coalescence of two compact objects. In the
first stage, the two objects orbit each other. The orbit slowly decays due to
energy loss into gravitational radiation. At the end of the inspiral phase, the
two objects plunge together. For black holes, this is the complicated merger
phase which is being studied numerically with fast supercomputers \cite{Mergers}.

At some point after the merger, the black hole settles down to the point where
it can be represented as a Kerr\cite{Kerr:BH} black hole undergoing
quasi-normal mode (QNM) oscillations \cite{Scott:RD}. This phase is called the ringdown
phase. The ringdown phase does not necessarily require a binary inspiral; any
perturbed Kerr black hole will ringdown through the emission of gravitational
waves. In this sense, the ringdown signal is one of the purest waveforms predicted
by General Relativity.

The most general stationary black hole metric is the 
Kerr-Newman metric\cite{BHWDNS}, which has only three free parameters: mass (M),
spin (J), and charge (Q). Two important special cases of this metric are
the Schwarzschild metric (charge = 0, spin = 0):

\begin{equation}
ds^2 = -\left(1 - \frac{2 \, G M}{c^2 r}\right) c^2 \, dt^2 
      + \left(1 - \frac{2 \, G M}{c^2 r}\right)^{-1} \, dr^2
      + r^2 \, d\theta^2 
      + r^2 \sin{\theta}^2 \, d\phi^2
\label{eq:Schwarz}
\end{equation}
and the Kerr metric (charge = 0):

\begin{equation}
ds^2 = -\left(1 - \frac{2 M r}{\Sigma}\right) c^2 \, dt^2 
       -\frac{4 a M r \sin{\theta}^2}{\Sigma} \, dt d\phi
       + \frac{\Sigma}{\Delta} \, dr^2
       + \Sigma \, d\theta^2 
       + \left(r^2 + a^2 + \frac{2 M r a^2 \sin{\theta}^2}{\Sigma} \right) \, d\phi^2
\label{eq:Kerr}
\end{equation}
where

\begin{equation}
a \equiv \frac{J}{M c}, \qquad  \Delta \equiv r^2 -\frac{2 \, G M r}{c^2} + a^2, 
\qquad \Sigma \equiv r^2 + a^2 \cos{\theta}^2  
\label{eq:subKerr}
\end{equation}
The theory of black hole perturbations and the associated radiation has a long history
~\cite{LivingReview:Ring}.
In 1957, Regge and Wheeler studied a perturbed Schwarzschild black hole
and found that it was stable to small perturbations~\cite{RW:RD}. In the 70s,
work by Chandrasekhar, Detweiler, Zerilli,and others analyzed perturbations of
Kerr black holes and the resulting gravitational waves. This work
showed that gravitational radiation from the quadrupole mode has the form of 
an exponentially damped sinusoid~\cite{Chandra:RD}. Approximate analytical 
expressions for
the central frequency (f) and the  quality factor (Q) are given 
in fits made by Echeverria~\cite{Echeverria:RD} to the numerical 
results of Leaver~\cite{Leaver:RD}:

\begin{equation}
f \simeq 32 \, \mbox{kHz} \left(\frac{M_{\astrosun}}{M}\right)
              \left[1 - 0.63 (1 - \Hat{a})^{3/10} \right]
\label{eq:RD_f}
\end{equation}

\begin{equation}
Q \simeq 2 (1 - \Hat{a})^{-9/20}
\label{eq:RD_Q}
\end{equation}
The ringdown waveform is \cite{Jolien:40m}:

\begin{equation}
h_{ave}(t) = A e^{-\pi f t / Q} \cos{(2 \pi f t)}
\label{eq:RD_h}
\end{equation}
where

\begin{equation}
A \simeq \frac{6 \times 10^{-21}}{\sqrt{Q (1 - 0.63 (1-\Hat{a})^{3/10})}}
         \left(\frac{\mbox{Mpc}}{r}\right) 
         \left(\frac{M}{M_{\astrosun}}\right)
         \left(\frac{\epsilon}{0.01}\right)^{1/2}
\label{eq:RD_A}
\end{equation}
is the amplitude, suitably averaged over spin-axis orientations and source
sky positions, $M$ is the mass of the black hole, $\epsilon$ is the fraction
of the black hole's rest mass which gets converted into gravitational radiation, and 
$\Hat{a} = (c/G) (J/M^2)$ is the
dimensionless spin parameter which goes from 0 (Schwarzschild) to 1 (extreme-Kerr).
It should be noted that this dimensionless $\Hat{a}$ is \emph{not} the same as the 
$a$ used in Equations~\ref{eq:Kerr} and \ref{eq:subKerr}.

For a 10 $M_{\astrosun}$ Schwarzschild black hole, if we take as a dynamical time 
the perimeter of the event horizon divided by the speed of light, we can also 
calculate a characteristic frequency, 
$f_{S} = (2 \pi R_{S}/c)^{-1} = c^3 / (4 \pi  G M) \simeq$ 1.6 kHz, which
is quite close to the estimate of 1.2 kHz from Equation~\ref{eq:RD_f}. The low
Q of 2  (from Equation~\ref{eq:RD_Q}) tells us that almost all of the energy is 
released in just a couple cycles.

Getting some physical intuition for the radiation from a Kerr black hole is
somewhat more difficult. An interpretation from Detweiler~\cite{Detweiler:Battelle}
is that in a spinning black hole, the metric perturbation from the pulsation
gets a frequency boost from the dragging of the inertial frame through which
it passes. As the hole approaches the extreme-Kerr limit 
($\hat{a} \rightsquigarrow 1$), the frequency of the wave as observed at infinity
gets shifted up to $\sim$2.7$\times$ the frequency of an equivalent mass
Schwarzschild black hole.

\subsubsection{Rate Estimates}

Given these formulae, we can estimate what black hole mass range is of interest for
LIGO. The lowest detectable black hole QNM frequency will be $\sim$50 Hz; this
corresponds to a 640 $M_{\astrosun}$ maximally spinning BH or a 240 $M_{\astrosun}$
BH with no spin. At the upper edge of the band, $\sim$5 kHz, we could detect a
6.4 $M_{\astrosun}$ BH with maximum spin or a 2.5 $M_{\astrosun}$ BH with no spin. 
This latter mass may result from the inspiral of the
1.4/1.4 $M_{\astrosun}$ NS/NS binaries.

Flanagan and Hughes~\cite{Scott:RD} estimate ringdown wave amplitudes
and SNR's in first-generation and advanced detectors. They optimistically
estimate an upper bound on the radiation efficiency of $\epsilon = 0.03$.

They estimate that the ringdown from a $\sim$100-700 $M_{\astrosun}$ BH, with
a Q of $\sim$12, would be seen with an SNR $>$ 10 at distances out to 100 Mpc
with a first generation LIGO interferometers. For these intermediate mass sources, 
the inspiral waveform would have too low of a frequency to be detectable.


\subsubsection{Previous Searches}

There have been two searches done to date for black hole ringdowns in 
the Milky Way:  one done by Creighton~\cite{Jolien:40m} using a single template 
on the data from the Caltech 40 m interferometer~\cite{40m:Inspiral}; 
more recently, the TAMA group~\cite{TAMA:RD} has conducted a search for ringdowns
using data from their 300 m interferometer during their Data Taking 6 run.

%------------------------------------------------------------------------------
