% ----- APPENDIX C ------------------%
%                                    %
%       The Mode Cleaner             %
% ---------------------------------- %
\chapter{Mode Cleaner}
\label{app:ModeCleaner}

\begin{figure}[!h]
\centerline{\includegraphics[angle=0,height=6in]{Figures/AppC/MC-block_diagram.png}}
\end{figure}

The Mode Cleaner (MC) is a suspended, in-vacuum, triangular cavity. Its
purpose is to condition the laser beam before it enters the
main interferometer.

Specifically, it serves three purposes:

\begin{itemize}
\item It is called a mode cleaner because it is resonant only for
      a single transverse spatial mode. Higher order spatial modes
      experience a larger phase shift per each round trip and so 
      fall out of the very narrow resonance. The effective frequency
      shift for the higher order modes is given by~\cite{Siegman}:

      \begin{equation}
      \Delta f_{mn} = (m + n + 1) \, \arccos{(g_{\mbox{\tiny MC}})} \, \frac{c}{2 \pi L}
      \end{equation}
      Since we know the transmission as a function of frequency, we can
      easily write down the amplitude transmission as a function of
      mode index:

      \begin{equation}
      t_{m n} = \frac{1-R}{1 + R^{2} - 2 R \cos{(m+n) \, 
                          \arccos{\sqrt{g_{\mbox{\tiny MC}}}}}}
      \label{eq:ModeCleaning}
      \end{equation}
 
\item The MC is also a very stable angular reference. Angular fluctuations of 
      the input beam can be represented as higher order transverse modes of the
      MC cavity basis. So the input beam jitter is passively filtered out.

\item The Mode Cleaner is used as a quiet length reference which the
      laser wavelength is compared to in an intermediate stage of the
      interferomter's Common Mode Servo. (See Figure~\ref{fig:CMblock})

\item The cavity passively filters laser frequency and amplitude
      fluctuations above the mode cleaner cavity pole, $f_{\mbox{\tiny MC}} \sim4$ kHz.
      The cavity's amplitude transmission response falls like
       $\sim$ 1/f for frequencies above the pole and below 
      the next resonance frequency.

\item Lastly, the mode cleaner is also a polarization filter. There
      is a 180 degree phase flip for horizontally polarized light
      relative to vertically polarized light. This phase shift comes
      from the triangular geometery of the cavity. Tracing the propagation
      of the E-field vector in a single round trip makes this effect clear.
      In addition to this purely geometric effect, there can be a different
      phase shift upon reflection for light of the two polarizations. This
      phase shift has not been measured yet. The power transmission of the
      flat mirrors (MC$_1$ and MC$_3$) is 10X greater ($\approx$ 2\%) for 
      the horizontal polarization than for the vertical and so the Finesse
      of the cavity for horizontally polarized light is 10X less.

\end{itemize}



\begin{table}[!h]
\begin{center}
\begin{tabular}{|l|c|r|c|}
\hline
\multicolumn{4}{|l|}
{{\bf Small Optic Parameters}}\\ \hline \hline
Parameter                          & Symbol     & Value           & Units   \\ \hline \hline

Thickness (coating)            & d$_C$      & 8 $\times$ 10$^{-6}$   & m           \\ \hline

Optic Diameter                        & dia        & 0.075           & m           \\ \hline

Optic Thickness                       & $h$        & 0.025           & m           \\ \hline

Optic Mass               & $m_{\mbox{\tiny SOS}}$  & 0.25            & kg          \\ \hline

\end{tabular}
\end{center}
\caption[Small Optic's Parameters]{Parameters for the optics are only approximate. Only
                            parameters which are different from the large optics 
                             (see Table~\ref{t:LOS} are listed. There is an
                             optic to optic variation in dimensions due to the varying
                             wedge angles. Thickness is measured at the \emph{thickest}
                             point for all small optics.}
\label{t:SOS}
\end{table}



\begin{table}[!h]
\begin{center}
\begin{tabular}{|l|c|r|c|}
\hline
\multicolumn{4}{|l|}
{{\bf Mode Cleaner Parameters}}\\ \hline \hline
Parameter                           & Symbol     & Value          & Units   \\ \hline \hline

Plane mirror transmittance   & T$_{MC1}$,T$_{MC3}$  & 2 $\times$ 10$^{-3}$   & -   \\ \hline

Curved mirror transmittance      & T$_{MC2}$     & 1 $\times$ 10$^{-5}$      & -   \\ \hline

Cavity length                    & L$_{MC}$     & 12.243           & m             \\ \hline

Free Spectral Range              & f$_{fsr}$    & 12.243           & MHz           \\ \hline

g - factor                       & g$_{MC}$     & 0.290            & -             \\ \hline

Finesse                   & $\mathcal{F}_{MC}$   & 1400            & -             \\ \hline

Cavity pole                 & f$_{MC}$           & 4000            & Hz            \\ \hline

\end{tabular}
\end{center}
\caption[Mode Cleaner Parameters]{Parameters for the suspended, in-vacuum Mode Cleaner
         cavity. The mirror transmittances, radii of curvature, and cavity g-factor are
         the designed parameters (not measured). All others are measured in-situ.}
\label{t:MCparams}
\end{table}


\section{Noise}
\label{sec:MCnoise}

The mode cleaner length fluctuations must be kept low enough to not compromise the
frequency noise stabilization. In addition, since the laser wavelength is stabilized
to the mode cleaner in one stage of the frequency stabilization system, the mode
cleaner length sensing noise must also be kept low.

As done for the main interferometer (in Chapter~\ref{chap:noise}), the noise budget
for the mode cleaner was made early on and used to motivate the design of the
electronics associated with the mode cleaner.

The old requirement curve in Figure~\ref{fig:MCnoise} is based on the initial 
interferometer design which
used a much more conservative frequency stabilization scheme. In the existing
situation, there is more gain in the final servo stage which stabilizes the
wavelength of the light transmitted by the mode cleaner to the average length of
the arms. The new requirement curve, then, is relaxed by the amount shown in the
plot. 

\begin{figure}[!h]
\centerline{\includegraphics[angle=0,width=6.5in]{Figures/AppC/mcn2.pdf}}
\caption[Mode Cleaner Noise Budget]{Frequency noise spectrum of the light
               transmitted by the mode cleaner. The true output noise is measured 
               by taking
               power spectra of the Common Mode Servo control signals and
               correcting appropriately for the CM loop gain.}
\label{fig:MCnoise}
\end{figure}

\subsection{Radiation Pressure}
Fluctuations in the stored power of the mode cleaner cause length fluctuations
through radiation pressure forces on the suspended mirrors. The effect is more
significant in the mode cleaner than in the main interferometer due to the
40$\times$ smaller mass of the mode cleaner mirrors. 

To calculate the radiation pressure effect on the MC round trip cavity length
we have to take into account the non-normal angle of incidence of the beam on
the flat mirror surfaces (the angle of incidence on the curved mirror, MC2, is
small enough to approximate as zero).

The change in the round trip length for small displacements of the MC mirrors
(in the direction perpendicular to their respective surfaces) is

\begin{equation}
\delta L_{\mbox{\tiny RT}} \simeq \frac{2}{\sqrt{2}} \delta x_{mc1} + 
                                  2 \delta x_{mc2} + 
                                  \frac{2}{\sqrt{2}} \delta x_{mc3}
\end{equation}

The displacements of the MC mirrors resulting from a fluctuating circulating power
are:

\begin{alignat}
\delta x_{mc1} &= \sqrt{2} \times \frac{\delta P_{circ}}{m \, c \, \omega^2} \\
\delta x_{mc2} &= 2 \times \frac{\delta P_{circ}}{m \, c \, \omega^2} \\
\delta x_{mc3} &= \sqrt{2} \times \frac{\delta P_{circ}}{m \, c \, \omega^2}
\end{alignat}

So then the corresponding change in the cavity length,
($L_{\mbox{\tiny MC}} \equiv L_{\mbox{\tiny RT}}/2$) is:

\begin{equation}
\begin{aligned}
\delta L_{\mbox{\tiny MC}} &= 4 \, \frac{\delta P_{circ}}{m c f^2} \\
        &\simeq 6 \times 10^{-19}
             \left(\frac{\mbox{RIN}(f)}{10^{-8} /\sqrt{\mbox{Hz}}}\right)
             \left(\frac{P_{in}}{1 W}\right)
             \left(\frac{\mathcal{F}}{1400}\right)
             \left(\frac{100}{f}\right)^2
             \frac{\mbox{m}}{\sqrt{\mbox{Hz}}}
\end{aligned}
\end{equation}


\subsection{VCO Phase Noise}
The voltage controlled oscillator (VCO) which drives the acousto-optic
modulator (AOM) used as a frequency shifter in the first stage of the
frequency stabilization servo (FSS) has a phase jitter associated with
the 80 MHz carrier with which the AOM is driven.

The phase noise is multiplied by 2 for the double pass of the beam through
the AOM. The resultant noise is incident on the mode cleaner and is suppressed
by the mode cleaner servo, giving it the shape seen in Figure~\ref{fig:MCnoise}.

The double-pass frequency noise induced on the light by the VCO has been measured
to be a flat level of $\sim20 \times 10^{-3} \mbox{Hz}/\sqrt{\mbox{Hz}}$.

From Figure~\ref{fig:CMnoise}, we see that this noise source is far above the
requirement for the MC's transmitted frequency noise. There are a few options
on this front:

\begin{itemize}
\item Lower noise VCO (probably at the cost of dynamic range).

\item Higher gain in the Mode Cleaner servo in the 3-10 kHz band.

\item Higher gain in the Common Mode servo in the 3-10 kHz band.
\end{itemize}


\subsection{Servo Electronics}
Due to excess noise coming from the laser, the signal entering the MC servo
board was too large and saturated the electronics (the op amps used had a 
low slew rate limit). Attenuating this signal heavily resulted in a high
effective sensing noise level in this servo. The induced noise is
suppressed somewhat by the servo loop gain.

This noise will be reduced by using faster, low-noise components and switching
the filtering configuration to one that preserves the input referred 
SNR up to 10 kHz.


\subsection{Acoustics/Clipping/Scattering}
Just like in the main IFO, there is noise introduced into the MC sensing
somewhere on the out-of-vaccuum optics table on which the MC sensing
optics and electronics are housed. There is evidence that this accounts for
the rich, unmodeled structure in the noise in the 100-1000 Hz band.

As shown in Figure~\ref{fig:CMnoise}, the frequency noise on the light
leaving the MC is too high above 100 Hz. In the 100-1000 Hz band, it
will be neccesary to pursue some mitigation of the acoustic sensitivity
of the MC sensing chain; namely, some sound dampening foam and better
alignment of the beam and expansion of some of the limiting optical apertures.













