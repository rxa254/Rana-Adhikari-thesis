%
%
%     INTRODUCTION
%
%
%------------------------------------------------------------------------------
\chapter*{Introduction}
\addcontentsline{toc}{chapter}{Introduction}

A handful of kilometer scale, laser interferometers have begun operation in the
last few years with the goal of detecting gravitational waves.
They are all steadily approaching their designed sensitivities, alternating data 
taking runs with further detector improvements.
This worldwide network of observatories includes the German-British 
GEO600~\footnote{
\href{http://www.geo600.uni-hannover.de}{http://www.geo600.uni-hannover.de}}
~\cite{GEO:StatusCQG}, 
the Japanese TAMA~\footnote{
\href{http://tamago.mtk.nao.ac.jp}{http://tamago.mtk.nao.ac.jp}}\cite{TAMA:StatusCQG},
and a set of 3 interferometers in the United States called 
LIGO~\footnote{
\href{http://ligo.caltech.edu}{http://ligo.caltech.edu}}
~\cite{LIGO:StatusCQG,Rai:PhysicsToday}. 
Also expected to come on-line in the near future is the Italian-French
VIRGO~\footnote{
\href{http://www.virgo.infn.it}{http://www.virgo.infn.it}}~\cite{VIRGO:StatusCQG}.
% and 
%the Australian AIGO~\footnote{
%\href{http://www.anu.edu.au/Physics/ACIGA}{http://www.anu.edu.au/Physics/ACIGA}}
%~\cite{ACIGA:StatusCQG}. 
All of these
observatories employ (or will employ) enhanced Michelson interferometers illuminated 
by highly stabilized, medium power lasers operating at 1064~nm. All of the 
interferometers' optics are suspended by seismic isolation systems and 
are housed in high to ultra-high vacuum beamtubes.

This thesis only describes the LIGO detectors, concentrating on the
Louisiana 4 km interferometer.

Chapter~\ref{chap:GW} describes briefly the generation of gravitational waves,
speculations on possible sources, and their detectability based on a theoretical
noise estimate of the interferometers.

Chapter~\ref{chap:IFO} motivates the design of the power recycled, 
Fabry-Perot Michelson interferometer configuration used in LIGO.

Chapter~\ref{chap:signals} describes the scheme
used to readout the signals conatining information about the interferometer
lengths and also the gravitational wave signal.

Chapter~\ref{chap:noise} lists all of the significant noise sources, their
coupling mechanisms, and makes estimates for their contribution to the total
noise budget. This chapter and the following one are the core of the thesis.

Chapter~\ref{chap:controls} discusses the control systems, mainly focusing
on the length controls: the motivation for controls, the troubles with their noise,
and some transfer functions of control loops. 

Chapter~\ref{chap:ringdowns} describes a search made through the data for
damped sinusoid signals.

Chapter~\ref{chap:future} gives examples of work that can be done on these first
generation of interferometers to dramatically increase the event rate. 

The Appendices provides some further details on topics which are briefly mentioned 
in the main text.


This work, and the LIGO Laboratory, is supported by the National 
Science Foundation\footnote{\href{http://www.nsf.gov/}{http://www.nsf.gov/}}, 
grant PHY-0107417.
