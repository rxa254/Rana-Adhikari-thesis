% ----- APPENDIX E ------------------%
%
%  Cavity formulas                   %
% ---------------------------------- %
\chapter{Cavity Formulas}
\label{app:modes}

A useful set of bassis functions for representing laser beams in cavities is in
terms of the Hermite-Gaussian modes \cite{Mizuno:Thesis}:

\begin{equation}
\begin{split}
\Psi_{mn}(x,y,z) =
                &\frac{1}{z+i z_{R}} \sqrt{\frac{2 z_{R}}{2^{m+n}m!n!\lambda}}
                H_{m}(\frac{\sqrt{2} x}{w(z)})H_{n}(\frac{\sqrt{2} y}{w(z)}) \\
         \times &exp[-i\frac{x^2+y^2}{z+i z_{R}}\frac{k}{2}+
                      i(m+n+1)arctan(\frac{z}{z_{R}})]
\end{split}
\end{equation}
Here the $H_{l}(\eta)$ are the Hermite polynomials of order $l$. A
characteristic scale at which the beam size has increased by $\sqrt{2}$
is $z_{R}$, the Rayleigh range and it is defined as 
$z_{R}=\pi w_{0}^{2}/\lambda$. The equation describing the expansion of
the beam is

\begin{equation}
w(z) = w_{0} \sqrt{1 + (z/z_{R})^{2}}
\end{equation}
where $w_{0}$ is the minimum in the beam size, similar to the focus in
geometric optics. $w(z)$ is defined as the 1/e point in field for a pure
Gaussian beam, i.e. the lowest order, TEM$_{00}$, Hermite Gaussian mode.

