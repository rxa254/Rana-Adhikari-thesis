% ---------- APPENDIX H ------------------%
%
%       The LSC Photodetector             %
% --------------------------------------- %
\chapter{The Photodetectors}
\label{app:RFPD}

\begin{figure}[!h]
\centerline{\includegraphics[angle=0,width=6.5in]{Figures/AppH/ASPD.pdf}}
\end{figure}

The photodiode in the photodetector is a an EG\&G Canada Ltd., C30642G InGaAs PIN
photodiode. It has a circular active area with a 2 mm diameter and is mounted in a 
TO-5 package with a built in glass window. The window has been removed with a 
can opener to reduce scatter losses from the glass surface.

\begin{figure}[!h]
\centerline{
\includegraphics[angle=0,width=6.5in]{Figures/AppH/PD-Circuit.png}}
\caption[Simplified Photodetector Schematic]{The yellow region is
         the equivalent photodiode. The L$_{\omega}$C$_D$ resonance made with
         the inductor in the orange region is tuned to the
         resonant sideband frequency ($f_m \approx$ 24.5 MHz). 
         The L$_{2 \omega}$-C$_{2 \omega}$
         notch is tuned to dump the 2 f$_m$ photocurrent. $e_n$ and $i_n$
         are the equivalent noise generators associated with the input of
         A$_{RF}$ (the MAX4107 RF Amplifier)}
\label{fig:RFPD}
\end{figure}

The photodetector circuit is shown in Figure~\ref{fig:RFPD}. The photodiode is modeled
as a current source in parallel with a capacitance. Not shown in the diagram are
circuits to read out the DC diode current and to dynamically adjust

The transimpedance, $Z(\omega)$, of the circuit is given by
\begin{equation}
\frac{1}{Z(\omega)} = \frac{1}{R_D + \frac{1}{i \omega C_D}}
                    + \frac{1}{i \omega L_{\omega} + R_{L \omega}}
                    + \frac{1}{R_{L 2 \omega} + \frac{1}{i \omega C_{2 \omega}} 
                    + i \omega L_{2 \omega}}
\end{equation}
The total voltage noise, e$_T$, referred to the non-inverting input of the amplifier is
\begin{equation}
e_T^2 = e_n^2 + i_n^2 Z_R^2 + 4 k_B T Z_R
\end{equation}
where $Z_R \equiv Z(\omega_m)$.


\section{AS\_I Servo}
\label{sec:Gorilla}

The path to lower noise above a few hundred Hz has always involved putting more light
on the photodetectors. Before the first science runs, it was noticed that the dynamic
range of the photodetector circuit was being exceeded with light levels $\sim$100X
smaller than the detectors were designed for.

This was due to a large signal at the anti-symmetric port in the orthogonal RF phase
(AS\_I) to the one which contains the gravity wave signal (AS\_Q). This AS\_I
signal produced a large RF current at $f_m$, saturating the MAX4107 pre-amplifier shown 
in Figure~\ref{fig:RFPD}. The mechanisms to generate the AS\_I signal are discussed 
in Section~\ref{sec:darksignals}.

There was no practical way to null this signal optically and so a circuit was designed to
null the signal electronically by adding in an RF current through a test input. 
The AS\_I is digitally filtered and used to drive the IF port of a double-balanced 
mixer (Mini-Circuits ZP-3MH). The LO port of the mixer was driven into saturation
by the same LO signal used to demodulate AS port RF signal. The RF port of the mixer
is then an RF signal, ampltiude modulated by the IF signal. The signal was sent through
a high power RF amplifier (Mini-Circuits ZHL-3A), a bandpass filter to reject harmonics
of the RF carrier and injected into the photodetector circuit through a step-up 
transformer as shown on the left hand side of Figure~\ref{fig:RFPD}.

At the time of this writing, the amount of detectable power at the AS port and
the high frequency, shot noise limited sensitivity of the interferometers is limited
by the size of the AS\_I signal. 



\begin{table}[!h]
\begin{center}
\begin{tabular}{|l|c|r|c|}
\hline
\multicolumn{4}{|l|}
{{\bf Photodiode Parameters}}\\ \hline \hline
Parameter                          & Symbol     & Value             & Units   \\ \hline \hline

Responsivity (@ 1064 nm)           & R          & 0.80              & A/W           \\ \hline

Quantum Efficiency                 & $\eta$     & 0.95              & -             \\ \hline

Active Area                        & A          & $\pi$             & mm$^2$        \\ \hline


Diode Capacitance (10 V Bias)    & C$_{\mbox{\tiny D}}$    & 75     & pF            \\ \hline
 
Diode Resistance             & R$_{\mbox{\tiny D}}$      & 11        & $\Omega$      \\ \hline

Tuned Inductance                   & L$_{\omega}$  & 0.4          & $\mu$H          \\ \hline

Tuned Inductor Resistance          & R$_{L \omega}$      & 1         & $\Omega$   \\ \hline


Trap Inductance                    & L$_{2 \omega}$      & 0.5        & $\mu$H     \\ \hline

Trap Capacitance                   & C$_{2 \omega}$      & 20            & pF       \\ \hline

Trap Inductor Resistance           & R$_{L 2 \omega}$    & 5         & $\Omega$   \\ \hline


RF Pre Amplifier             & A$_{\mbox{\tiny RF}}$    & MAX4107          & -      \\ \hline

MAX4107 Voltage Noise @ 100 Hz     & e$_n$     & 1.3     & nV/$\sqrt{\mbox{Hz}}$    \\ \hline 

MAX4107 Current Noise @ 100 Hz     & i$_n$     & 4     & pA/$\sqrt{\mbox{Hz}}$      \\ \hline 

AS\_I Servo Resistor         & R$_{\mbox{\tiny ASI}}$   & 500  & $\Omega$           \\ \hline

\end{tabular}
\end{center}
\caption[Nominal PD Parameters]{These are the nominal parameters for the photodetector.}
\label{table:PDparams}
\end{table}

