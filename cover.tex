% -*-latex-*-
%
%---------  COVER PAGE
%  --------     &
% -----  ACKNOWLEDGEMENTS
%

\title{Sensitivity and Noise Analysis of 4 km Laser Interferometric Gravitational Wave Antennae}

\author{Rana Adhikari}
\department{Department of Physics}

\degree{Doctor of Philosophy}
\degreemonth{July}
\degreeyear{2004}
\thesisdate{July 7, 2004}

%% By default, the thesis will be copyrighted to MIT.  If you need to copyright
%% the thesis to yourself, just specify the `vi' documentclass option.  If for
%% some reason you want to exactly specify the copyright notice text, you can
%% use the \copyrightnoticetext command.  
%\copyrightnoticetext{\copyright IBM, 1990.  Do not open till Xmas.}

% If there is more than one supervisor, use the \supervisor command
% once for each.
\supervisor{Rainer Weiss}{Professor}
\supervisor[Thesis Co-Supervisor]{Peter Fritschel}{Principal Research Scientist}
%\supervisor{Peter Fritschel}{Research Scientist}

% This is the department committee chairman, not the thesis committee
% chairman.  You should replace this with your Department's Committee
% Chairman. Except for in the physics department where we use
% Greytak apparently
\chairman{Thomas J. Greytak}{Associate Department Head for Education}

% Make the titlepage based on the above information.  If you need
% something special and can't use the standard form, you can specify
% the exact text of the titlepage yourself.  Put it in a titlepage
% environment and leave blank lines where you want vertical space.
% The spaces will be adjusted to fill the entire page.  The dotted
% lines for the signatures are made with the \signature command.
\maketitle

% The abstractpage environment sets up everything on the page except
% the text itself.  The title and other header material are put at the
% top of the page, and the supervisors are listed at the bottom.  A
% new page is begun both before and after.  Of course, an abstract may
% be more than one page itself.  If you need more control over the
% format of the page, you can use the abstract environment, which puts
% the word "Abstract" at the beginning and single spaces its text.

%% You can either \input (*not* \include) your abstract file, or you can put
%% the text of the abstract directly between the \begin{abstractpage} and
%% \end{abstractpage} commands.

% First copy: start a new page, and save the page number.
\cleardoublepage
% Uncomment the next line if you do NOT want a page number on your
% abstract and acknowledgments pages.
\pagestyle{empty}
\setcounter{savepage}{\thepage}
\begin{abstractpage}
% $Log: abstract.tex,v $
%             
%              ABSTRACT
% 
%

Around the world, efforts are underway to commission several kilometer-scale 
laser interferometers to detect gravitational radiation. In the United States,
there are two collocated interferometers in Hanford, Washington and one interferometer
in Livingston, Louisiana. Together, these three interferometers form the Laser
Interferometric Gravitational-wave Observatory (LIGO). 

The core of the work described in this thesis is the modeling and reduction of
the noise in the interferometers which limits their ultimate sensitivity.

A vital component of the noise reduction is the modeling, design, and implementation
of $\sim$100 feedback control systems. The most critical of these systems
are described and motivated.

Although improvements are continuously being made to the stability and noise character
of these detectors, several months of data have been collected. Various efforts
are underway to search through these data for gravitational wave signals.
Included here, is a description of a search made through the data for signals
from the ringdown of the quasi-normal modes of Kerr black holes.

In addition, several possible future improvements to the detectors are outlined.

\end{abstractpage}

% Additional copy: start a new page, and reset the page number.  This way,
% the second copy of the abstract is not counted as separate pages.
% Uncomment the next 6 lines if you need two copies of the abstract
% page.
% \setcounter{page}{\thesavepage}
% \begin{abstractpage}
% % $Log: abstract.tex,v $
%             
%              ABSTRACT
% 
%

Around the world, efforts are underway to commission several kilometer-scale 
laser interferometers to detect gravitational radiation. In the United States,
there are two collocated interferometers in Hanford, Washington and one interferometer
in Livingston, Louisiana. Together, these three interferometers form the Laser
Interferometric Gravitational-wave Observatory (LIGO). 

The core of the work described in this thesis is the modeling and reduction of
the noise in the interferometers which limits their ultimate sensitivity.

A vital component of the noise reduction is the modeling, design, and implementation
of $\sim$100 feedback control systems. The most critical of these systems
are described and motivated.

Although improvements are continuously being made to the stability and noise character
of these detectors, several months of data have been collected. Various efforts
are underway to search through these data for gravitational wave signals.
Included here, is a description of a search made through the data for signals
from the ringdown of the quasi-normal modes of Kerr black holes.

In addition, several possible future improvements to the detectors are outlined.

% \end{abstractpage}

\cleardoublepage

\section*{Acknowledgments}

I have had the uncommon luck of working with many people in the project.
Its probably true that I've learned something from each and so a complete
list of people I would thank would make this thesis so thick that even
fewer people would read it. Instead I will just thank the people who
have given me the most work to do.

To Mike, thanks for teaching me how to drag wipe optics and how 
to calibrate a scope probe. Thanks for yelling at me if I (or any other grad
student) made a mess in the machine shop. Thanks for all the free food
and for taking that lemon off of my hands.

To Peter, thanks for putting up with all of those bad measurements,
ideas, and electronics. And for that timely invitation to
help out with that 15 meter cavity.

To Rai, thanks for hiring me. This is the longest I've ever held a job.
Thank you for letting me practice with one of your interferometers for a
few years. And most of all for being an example of the integrity and the
passion with which science should be done.

-

- Rana \today
%%%%%%%%%%%%%%%%%%%%%%%%%%%%%%%%%%%%%%%%%%%%%%%%%%%%%%%%%%%%%%%%%%%%%%
% -*-latex-*-
