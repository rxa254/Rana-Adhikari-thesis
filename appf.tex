% ------------------ APPENDIX F ------------------%
%                                                 %
%      Optics Parameters Characterizations        %
% ----------------------------------------------- %
\chapter{Characterization of the Optical Parameters}
\label{app:optics}



This Appendix describes characterization of many of the optical parameters of the
interferometer, including lengths, losses, reflectivities and
the cavity time constants and build-up factors.


\section{Cavity Length Measurements}

There are 4 primary lengths which are controlled. In addition to the microscopic
lengths, the macroscopic absolute lengths of these degrees of freedom must also
be set. Typically, this is done to the accuracy of the initial surveying techniques.

After the interferometer was running we measured these lengths interferometrically
and then adjusted the lengths accordingly.


\subsection{Arm Cavity Lengths}

The nominal arm lengths are 3995.15 meters. This was initially set by reference
to the GPS satellites~\cite{Rai:GPS}. We then determined the arm lengths 
inteferometrically by measuring the frequency between successive resonances. We
did this by applying a frequency modulation to the laser, generating a set of
frequency sidebands on the carrier field. We then did a swept sine measurement
between the laser excitation and the response in the reflection locking signal
for each arm individually. Measurements made in November of 2003 for L1 gave: \\

L$_x$ = 3995.032 m  $\qquad$  L$_y$ = 3995.001 m  \\

This took place after the translation of ITMY to correct the lengths of the
power-recycled Michelson. The ITMY was moved 4 cm closer to beamsplitter and
so lengthened the y-arm. This appears to imply that the arm lengths were 
$\approx$ 7 cm different before the summer of 2003 (which includes S1 \& S2).

Even more accurate measurements of arm length were made on the Hanford 4 km
interferometer \cite{Rick:ArmLengths} using a more sophisticated optical model
of the interferometer and a rubidium frequency standard. This resulted in
arm length measurements with a precision of 80 $\mu$m, which is less than the
daily stretching of the arm by the tidal gravitational force from the moon.


\subsection{Schnupp Asymmetry}

The macroscopic difference in the two Michelson lengths $l_y$ and $l_x$ determines
the transmission of the resonant sidebands to the dark port: 
$t_{\mbox{\tiny M}} = sin(\frac{2 \pi f_m l_-}{c})$.

This length difference is measured by finding the dark port demodulation phase 
appropriate for each arm cavity seperately. The difference in RF phase for
the two arms comes only from the different path length traveled by the sideband
to the AS port.

Typically, these measurements achieve an absolute accuracy of +/- 1 mm in this
length. There have not been significant efforts to improve on this accuracy and
there does not seem to be any reason to do so.

There was an error in the initial calculation to set this length for both 4 km
interferometers, with the result that the asymmetry was to small by $\sim$4 cm.
This was corrected by venting the corner station vacuum and moving the optics'
suspension towers in the summer of 2003 between the S2 \& S3 runs.


\subsection{Recycling Cavity Length}

The average length between the power recycling mirror (RM) and the two
input test masses (ITMs) is defined as the macroscopic length, $l_+$.
It is chosen to make the recycling cavity resonant for the resonant
sidebands (hence the name).

Hypothetically, a gross error in the placement of one of the mirrors
(e.g. the RM) would shift the resonant frequency of the cavity. This would
result in a reduced build up for both the carrier and the sidebands. Two
methods have been used to measure this length:

The first method was to misalign the RM, ETMX, and ETMY, forming a simple
Michelson interferometer. The differential Michelson length signal at the
AS port then appears in only one demodulation quadrature. This phase is
recorded and then measured again when the full interferometer is locked by
driving the differential arm length. 

The difference in the RF phase between
the two states gives a measure of how much the RF sidebands have been phase
shifted upon transmission through the resonant power-recycling cavity.
If the power recycling cavity length ($l_+$) is not an integer multiple
of $c/(2 f_m)$, there will be a differential phase shift applied to
the upper and lower RF sidebands as they transmit to the AS port.

%How much? Is this really true? Should there be a phasor diagram here?


\subsection{Mode Cleaner Length}

The mode cleaner length is set to be resonant for the interferometer's
resonant sideband. The non-resonant sideband frequency, f$_{\mbox{\tiny NR}}$, is then
set to be a multiple of the mode cleaner free spectral range
(see Table~\ref{t:MCparams}), but not resonant in the power recycling
cavity.

%If the MC length is not set correctly there can be some problems: such as...


\section{Transmissivity, Reflectivity, and Loss}

None of the optics have exactly the same transmissivity as was asked for 
or measured in the metrology lab, but they're pretty close mostly. The
AR coating reflectivities seem to be as much as 10X different than their
lab measured values.

While disturbing, we can use this to our advantage by using the most
powerful pickoff in the control loops for the $l_-$ and $l_+$
lengths.

\subsection{Common arm loss via recycling gain}

The carrier power recycling gain is given by the following formula:

\begin{equation}
G_{cr} = |g_{cr}|^2 
       = \left [\frac{t_{\mbox{\tiny RM}}}{1+r_{\mbox{\tiny RM}}r_c} \right ]^2
\end{equation}
where $r_c$ is the average amplitude reflectivity of the arms for the carrier. By
modifying Equation~\ref{eq:arm_reflectivity} to include a finite loss upon reflection
from the arm cavity mirrors we get:

\begin{equation}
r_c = \frac{r_{\mbox{\tiny ITM}} - \sqrt{1-T_{\mbox{\tiny ETM}}-L}}
           {1 - r_{\mbox{\tiny ITM}} \sqrt{1-T_{\mbox{\tiny ETM}}-L}}
\label{eq:arm_reflectivity_with_loss}
\end{equation}
which gives the handy approximate relation $(1-r_c) \approx 1\% (L / 140 ppm)$
for small losses. Here $L$ is defined as the round trip \emph{power} loss
in the arm cavity, including the ETM transmission, but not the ITM
transmission.

The measured power recycling gain factor of $\approx$50 gives us an average
arm loss of 140 ppm, or 70 ppm per mirror.


\subsection{Differential arm loss via PRC to AS\_Q \& CMRR}

Several pernicious noise sources highlighted in Chapter~\ref{chap:noise}
are proportional to the amplitude of the TEM$_{00}$ carrier field at the
AS port which is in the orthogonal phase from the carrier field produced
by a differential arm length offset.

Another way to describe this is that a differential arm length shift
causes the carrier fields interfering at the beamsplitter to have a relative
phase shift. The AS\_Q signal is \emph{not} first order sensitive to an
amplitude difference in the two fields, and so the $L_-$ servo does not
null this component of the carrier field. This amplitude unbalance can
only come about through a difference in the resonant reflectivity of the
arms.

As described above, the measurement of the cavity reflectivity gives a 
measure of the loss in the arm. By calibrating the frequency noise
coupling and the $l_+ \rightsquigarrow AS\_Q$ coupling (see Sections
\ref{sec:FreqNoise} and \ref{sec:POBnoise}, respectively) we know that 
the difference in
the resonant reflectivity of the two arms, $\delta r_c$ is 0.5\%. This
corresponds to a differential loss of 70 ppm between the two arms for the
Livingston interferometer. Similar measurements done on the Hanford 4 km 
interferometer give roughly
the same average loss but a factor of $\sim$2 less differential loss.

\subsection{T$_{\mbox{\tiny ITM}}$}

The transmission of the ITM ($\approx$ 0.028) is the dominant loss in 
the arm cavities. To determine the exact number, we measure the decay
time of the arm cavities by quickly switching off the light incident
on the interferometer. 

The best way to quickly shut off the light is to very quickly shift
the laser frequency with the FSS by electronically changing the sign
of the servo~\cite{Andri:elog}. The switch off time is then dominated by the time
constant of the in-vacuum Mode Cleaner ($\approx$20 $\mu$s decay time). 



\subsection{T$_{\mbox{\tiny ETM}}$}

T$_{\mbox{\tiny ETM}}$ is measured by directly measuring the power
transmitted through the end of the arms with the interferometer
locked. The circulating power in the arm is measured through
two methods: by using the known arm cavity build up factor and
the carrier recycling gain or by looking at the radiation
pressure offset induced on the ITMs (which are free to move at DC).


\subsection{R$_{\mbox{\tiny BS}}$ - T$_{\mbox{\tiny BS}}$}

Almost all of the calculations so far in this thesis have assumed that the
Beamplitter is truly a 50/50 beamsplitter; that
R$_{\mbox{\tiny BS}}$ = T$_{\mbox{\tiny BS}}$. One known consequence of
this, an increased radiation pressure coupling, was discussed in
Section~\ref{sec:radpress}.

There are no first order offsets introduced at the AS port from this
effect, however. Neglecting radiation pressure, all of the fields at the
AS port depend on the product (r$_{\mbox{\tiny BS}}$ t$_{\mbox{\tiny BS}}$)
and not on the difference.

\section{Contrast}

To have good sensitivity at high frequencies, the interferometer must be
shot noise limited. The shot noise limit must also be low to have a good
sensitivity, as described in Section~\ref{sec:ShotNoise}.

The contrast measurements to date have been made by measuring the
recycling gain for the carrier, the sideband power at the AS port, and then
the total power at the AS port. The contrast defect, $c_d$, is then
defined as:

\begin{equation}
c_d = \frac{P_{AS}}{P_{BS}} = \frac{P_{AS}}{G_{cr} P_{in}}
\end{equation}
where $P_{BS}$ is the total carrier power on the beamplitter and
$P_{AS}$ is the total carrier power at the anti-symmetric port.

The contrast defect in the Livingston interferometer has been measured to be
$c_d < 4 \times 10^{-5}$. This is an upper limit; the angular
fluctuations of the interferometer kept the dark port power fluctuating
by an order of magnitude over minute time scales.


\section{Metrology}

All of the optics metrology data is available at Core Optics 
website~\cite{COCwebsite} and interesting related comments in
an internal LIGO technical report~\cite{Rai:Metrology}.

